
\documentclass[titlepage]{article}

\begin{document}

\section{HKStack Complexity}

A problem with the HK stacking method is the fact that multiple maxima may occur. Figure XXX
\marginpar{\fbox{stack figure of BK.CMB}} illustrates this for station BK.CMB. The maxima at 37 km most likely corresponds to the crust, even though it is slightly smaller in amplitude than the global maximum at 56 km. In the absence of outside knowledge of the are it is difficult to automatically determine when a smaller local maximum is more appropriate than the global maximum, in general this judgement needs to be made by an analyst. However, it is useful to have some quantitative measures that give information as to the believability of the result and if the global maxima is a good representation of the true crustal structure.

One measure of the goodness of the maximum of the HK stack is how much of the power in the stack is due to the global maxima. If one maxima far exceeds any other local maxima, then in the absence of poor quality receiver functions, one is forced to either accept it as representing the crust, or argue that the moho does not produce a strong signal for this station. We introduce the complexity of the HK stack as the ratio of the power in the residual HK plot to the power in the original stack. The residual is the difference between the original stack and an HK stack based on synthetic receiver functions for a crustal model that corresponds to the global maxima. The synthetic HK stack is very insensitive to ray parameter, and hence epicentral distance, and so there is no need to compute the synthetics at distances corresponding to all the earthquakes used in the stack, nor even a range of distances. 

Bootstrap errors

Multiple maxima also contribute to large errors in the bootstrap. If there are several maxima whose amplitudes are reasonably close, then within each bootstrap iteration there is a reasonable probability that one besides the global maxima will be the maximum. If this does happen frequently in the bootstrap, then the variance will be large, giving a second indication that the result is not to be trusted without further review.

Ratio of global max to next largest local maxima

A similar measure is the ratio of the amplitude of the global maximum to the next largest local maxima. If the global maximum is significantly larger then any other local maxima, then it is more likely that the resulting value can be trusted.

Comparison to prior results

Of course, prior results are highly useful in determining if the result if believable. But prior results are have their own errors and inconsistencies, most significantly that they do not all exist at all stations. Crust2.0 is certainly useful in this regard because there is a result for the entire earth. However, there can be significant variation over the 2 degree grid size in their model.

\end{section}
\end{document}